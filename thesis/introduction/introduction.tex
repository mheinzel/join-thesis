Process calculi allow a formal, abstract description of concurrent
and distributed systems, to simplify mathematical reasoning.
In this setting, we are often interested in the
\emph{degree of distributability} of a term,
which can be intuitively describes as the number of parallel components
that can execute independently.
However, when implementing these calculi in practice,
it is often not possible to preserve their degree of distributability.
In \picalc, for example,
the problem arises from the possible existence of multiple senders
and multiple receivers on a channel.
A process wanting to send a message to one of many possible receivers
has to make sure the receiver did not already receive a messages
from somewhere else in the meantime.
In this way, it can become necessary to solve a global consensus problem
for the correct execution of basic communication primitives.

As presented by Peters et al. \cite{peters_distributability_2013},
this lack of distributability in \asyncpicalc can be shown by the existence of
certain synchonization patterns.
At the same time, the \emph{join-calculus} \cite{fournet_reflexive_1996}
is identified as a calculus that is more distributable.
And indeed, one of its main ideas is to enforce that all receivers on a channel
reside in the same location (\emph{locality}).
It also explores the addition of a notion of locations,
which allows it to not only be \emph{distributable},
but also express explicitly \emph{distributed} processes
and migration \cite{fournet_calculus_1996}.
With these features, it is well-suited to be a practical framework for
writing distributed and concurrent programs.

There exist several implementations of the concept of join-patterns
\footnote{\url{https://www.microsoft.com/en-us/research/project/comega/}}
\footnote{\url{https://github.com/Chymyst/chymyst-core}},
but the only one making use of the distribution primitives
is an unmaintained beta version of
JoCaml\footnote{\url{http://moscova.inria.fr/oldjocaml/index.shtml}}
that makes heavy modifications to the original OCaml compiler and runtime system
\cite{conchon_jocaml:_1999}.
This sacrifices binary compatibility and makes it difficult to incorporate
upstream changes.
Later versions of JoCaml\footnote{\url{http://jocaml.inria.fr/}}
dropped support for locations and related primitives.

If it was possible to rely on existing infrastructure,
much effort could be avoided.
For instance, there are multiple implementations of the \actormodel
providing support for distributed communication and code migration.
Also, actors are freely distributable, which should make it possible
to find a distributability-preserving translation from \joincalc.
This thesis aims to find such an encoding and, going from there,
develop a prototype implementation of \distjoincalc.
