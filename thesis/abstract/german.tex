\begin{abstract}
  Der Join-Kalkül ist eine praktisch verteilbare Variante des Pi-Kalküls
  und besitzt eine verteilte Version mit expliziten Lokalitäten und Migration.
  Auch wenn diese Eigenschaften vielversprechend für den Einsatz in einem
  praxis-orientierten Umfeld sind,
  erfordert die Implementierung von Features wie maschinen-übergreifender
  Kommunikation und Migration einen hohen Aufwand.
  Daher untersuchen wir die Möglichkeit,
  existierende Infrastruktur in aktorenbasierten Systemen zu verwenden.

  Wir entwickeln eine verteilbarkeitserhaltende Übersetzung des
  vereinfachten Core-Join-Kalküls in den Kalkül \actorpicalc
  (einen typisierten Pi-Kalkül, der das Aktorenmodell repräsentiert)
  und leiten daraus die Implementierungen
  einer Einbettung des Join-Kalküls in die aktorenbasierte Programmiersprache
  Erlang ab.
  Der Prototyp wird anschließend um die Primitive des
  Verteilten Join-Kalküls, einschließlich Migration, erweitert.
\end{abstract}
