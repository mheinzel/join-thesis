\begin{abstract}
  Der Join-Kalkül ist eine besser verteilbare Variante des Pi-Kalküls
  und besitzt eine verteilte Erweiterung mit expliziten Lokalitäten und Migration.
  Auch wenn diese Eigenschaften vielversprechend für den Einsatz in einem
  praxis-orientierten Umfeld sind,
  erfordert die Implementierung von Features wie maschinen-übergreifender
  Kommunikation und Migration einen hohen Aufwand.
  Daher untersuchen wir die Möglichkeit,
  existierende Infrastruktur in aktorenbasierten Systemen zu verwenden.

  Wir entwickeln eine verteilbarkeitserhaltende Übersetzung des
  vereinfachten Core-Join-Kalküls in den Kalkül \actorpicalc
  (einen typisierten Pi-Kalkül, der das Aktorenmodell repräsentiert)
  und erstellen daraus die Implementierungen
  des Join-Kalküls in der aktorenbasierte Programmiersprache
  Erlang.
  Der Prototyp wird anschließend um die Primitive des
  Verteilten Join-Kalküls, einschließlich Migration, erweitert.
  Durch die Nutzung von Erlang/OTP als Grundlage
  erhalten wir eine überschaubare und einfache Implementierung.
\end{abstract}
