\section{The \JoinCalc}

% short introduction?
% or in thesis introduction?

% grammar

We use \joincalc as described by Fournet et al.
\cite{fournet_calculus_1996}.
Assuming an infinite set of channel names
$ \mathcal{N} = \{ x, y, z, u, v, \kappa, \ldots \} $,
we use
$ \tilde{x}, \tilde{y}, \ldots $
for tuples of names
and define the syntax as follows.

% turn into figure again?
% left-align explanations?
\begin{align}
  \tag*{\textbf{processes}}
  \\
  P, Q
  \grmr \jsnd{x}{\tilde{v}}   \tag*{emission of $\tilde{v}$ on $x$}
  \altn \nonumber \jdef{D}{P} \tag*{definition of D in P}
  \altn P \jpar Q             \tag*{parallel composition}
  \altn \jnullproc            \tag*{empty process}
  \\ \nonumber
  \\
  \tag*{\textbf{definitions}}
  \\
  D, E
  \grmr \jrct{J}{P} \tag*{elementary clause}
  \altn D \jcon E   \tag*{simultaneous definitions}
  \altn \jnulldef   \tag*{empty definition}
  \\ \nonumber
  \\
  \tag*{\textbf{join-patterns}}
  \\
  J, K
  \grmr \jrcv{x}{\tilde{v}} \tag*{reception of $\tilde{v}$ on x}
  \altn J \jpat K           \tag*{composed join-pattern}
\end{align}

Readers familiar with \asyncpicalc will recognize emission,
parallel composition and the empty process, but miss the usual constructs for
defining new channels, reception and replication.
All these concerns are covered by a single primitive,
$ \jdef{D}{P} $.

% TODO: explanations?
% TODO: scopes, defined names?
% check literature

Note that all received names in a join-pattern must be distinct,
making both patterns
$ \jrcv{x}{u,u} $ and
$ \jrcv{x}{u} \jpat \jrcv{y}{u} $
invalid.


\begin{alignat*}{2}
  & \rv{\jrcv{x}{\tilde{v}}} & &= \{ u \in \tilde{v} \} \\
  & \rv{J \jpat K}           & &= \rv{J} \uplus \rv{K} \\
  \\
  & \dv{\jrcv{x}{\tilde{v}}} & &= \{ x \} \\
  & \dv{J \jpat K}           & &= \dv{J} \cup \dv{K} \\
  \\
  & \dv{\jrct{J}{P}}         & &= \dv{J} \\
  & \dv{D \jcon E}           & &= \dv{D} \cup \dv{E} \\
  & \dv{\jnulldef}           & &= \emptyset \\
  \\
  & \fv{\jrct{J}{P}}         & &= \dv{J} \cup (\fv{P} - \rv{J}) \\
  & \fv{D \jcon E}           & &= \fv{D} \cup \fv{E} \\
  & \fv{\jnulldef}           & &= \emptyset \\
  \\
  & \fv{\jsnd{x}{v}}         & &= \{ x \} \cup \{ u \in \tilde{v} \} \\
  & \fv{\jdef{D}{P}}         & &= (\fv{P} \cup \fv{D}) - \dv{D} \\
  & \fv{P \jpat Q}           & &= \fv{P} \cup \fv{Q} \\
  & \fv{\jnullproc}          & &= \emptyset \\
\end{alignat*}


% semantics

% RCHAM
% \cite{berry_chemical_1990}?
The semantics of \joincalc is described using the reflexive chemical abstract
machine (RCHAM) \cite{fournet_reflexive_1996}.
A RCHAM configuration
$ \rchamcnf{\mathcal{R}}{\mathcal{M}} $
consists of a multiset of active definitions $\mathcal{R}$,
called ``reactions",
and a multiset of active processes $\mathcal{M}$,
called ``molecules".

There are multiple reversible heating/cooling rules $\rchamheating$,
corresponding to the structural congruence between processes,
and a single reduction rule $\rchamreduction$.
In each of these rules, there is an implicit unchanged context of reactions and
molecules which we omit for brevity.

\begin{align*}
  \rchamequ{str-join}{
    \rchamcnf*{}{P \jpar Q}
  }{
    \rchamcnf*{}{P, Q}
  }
  \\
  \rchamequ{str-null}{
    \rchamcnf*{}{\jnullproc}
  }{
    \rchamcnf*{}{}
  }
  \\
  \rchamequ{str-and}{
    \rchamcnf*{D \jcon E}{}
  }{
    \rchamcnf*{D, E}{}
  }
  \\
  \rchamequ{str-nodef}{
    \rchamcnf*{\jnulldef}{}
  }{
    \rchamcnf*{}{}
  }
  \\
  \rchamequ{str-def}{
    \rchamcnf*{}{\jdef{D}{P}}
  }{
    \rchamcnf*{D\sigma_{\opdv}}{P\sigma_{\opdv}}
  }
  \\
  \\
  \rchamred{red}{
    \rchamcnf*{\jrct{J}{P}}{J\sigma_{\oprv}}
  }{
    \rchamcnf*{\jrct{J}{P}}{P\sigma_{\oprv}}
  }
\end{align*}

In (str-dev), the substitution $\sigma_{\opdv}$ instantiates the defined
variables $\dv{D}$ to distinct, fresh names that are not free in any reaction
or molecule in the configuration.
In (red), $\sigma_{\oprv}$ simply substitutes the received variables $\rv{J}$.
Note that (red) makes use of the syntactic similarity between join-patterns
and parallel emission of messages in processes.

% explanations?
% note replication etc?

To obtain a reduction relation $\joinreduction$
on processes (instead of RCHAM configurations),
we can define for any two processes P and P':
\begin{equation*}
  P \joinreduction P'
  \ \ \textbf{iff} \ \ 
  \rchamcnf{\emptyset}{P}
  \rchamheating^*\rchamreduction\rchamheating^*
  \rchamcnf{\emptyset}{P'}
\end{equation*}

% macro for examples?
Let us give an example:

\begin{align*}
  &&\rchamcnfset*
    {}
    {\jdef
      {\jrct{\jrcv{x}{u} \jpat \jrcv{y}{v}}{P}}
      {\jsnd{x}{a} \jpar \jsnd{x}{b} \jpar \jsnd{y}{c}}}
  \\
  \rchamheating\ \tag{str-def}
  &&\rchamcnfset*
    {\jrct{\jrcv{x}{u} \jpat \jrcv{y}{v}}{P}}
    {\jsnd{x}{a} \jpar \jsnd{x}{b} \jpar \jsnd{y}{c}}
  \\
  \rchamheating\ \tag{str-join}
  &&\rchamcnfset*
    {\jrct{\jrcv{x}{u} \jpat \jrcv{y}{v}}{P}}
    {\jsnd{x}{a}, \jsnd{x}{b} \jpar \jsnd{y}{c}}
  \\
  \rchamreduction \tag{red}
  &&\rchamcnfset*
    {\jrct{\jrcv{x}{u} \jpat \jrcv{y}{v}}{P}}
    {\jsnd{x}{a}, P\substitution{\subst{b}{u}, \subst{c}{v}}}
  \\
  \rchamheating\ \tag{str-join}
  &&\rchamcnfset*
    {\jrct{\jrcv{x}{u} \jpat \jrcv{y}{v}}{P}}
    {\jsnd{x}{a} \jpar P\substitution{\subst{b}{u}, \subst{c}{v}}}
  \\
  \rchamheating\ \tag{str-def}
  &&\rchamcnfset*
    {}
    {\jdef
      {\jrct{\jrcv{x}{u} \jpat \jrcv{y}{v}}{P}}
      {\jsnd{x}{a} \jpar P\substitution{\subst{b}{u}, \subst{c}{v}}}}
\end{align*}

and thus

\begin{equation*}
  \jdef
    {\jrct{\jrcv{x}{u} \jpat \jrcv{y}{v}}{P}}
    {\jsnd{x}{a} \jpar \jsnd{x}{b} \jpar \jsnd{y}{c}}
  \ \joinreduction\ 
  \jdef
    {\jrct{\jrcv{x}{u} \jpat \jrcv{y}{v}}{P}}
    {\jsnd{x}{a} \jpar P\substitution{\subst{b}{u}, \subst{c}{v}}}
\end{equation*}

Note the non-determinism in receiving $b$ instead of $a$ on $x$.
Instead of
$ \jsnd{x}{a} \jpar P\substitution{\subst{b}{u}, \subst{c}{v}} $
we could also arrive at
$ \jsnd{x}{b} \jpar P\substitution{\subst{a}{u}, \subst{c}{v}} $.


\subsection{The \CoreJoinCalc}

The full calculus can be encoded in a fragment of itself, called core calculus
\cite{fournet_reflexive_1996}.
Since all of its processes are also valid processes in the full calculus,
it suffices to describe its semantics using the RCHAM as shown above,
only using the rules (str-join), (str-def) and (red).

\begin{JDef}{grammar of \corejoincalc}
  P, Q
  \grmr \jsnd{x}{u}
  \altn P \jpar Q
  \altn \jdef{\jrct{\jrcv{x}{u} \jpat \jrcv{y}{v}}{P}}{Q}
\end{JDef}


\subsection{The \DistJoinCalc}

% TODO: short explanation of idea
\cite{fournet_calculus_1996}


In addition to an infinite set of channel names
$ \mathcal{N} = \{ x, y, z, \kappa, \ldots \} $,
we now use an infinite set of location names
$ \mathcal{L} = \{ a, b, c, \ldots \} $.
These can also be sent and received, which means that values
$ u, v, \ldots $ in $ \jsnd{x}{\tilde{v}} $
can now be either channel or location names.

% TODO: not a figure
\begin{JDef}{grammar of \distjoincalc}
  P, Q
  \grmr \dots
  \altn \jgo{b}{\kappa}
  \\ \\
  D, E
  \grmr \dots
  \altn \jloc{a}{D}{P}
\end{JDef}

% TODO: explain what new constructs mean?
Also, we demand two additional syntactic restrictions:
A location $a$ can only be defined once in any definition $D$.
A defined channel $x$ may only appear in the join-patterns of one location.
These conditions make the following definitions invalid:

\begin{align*}
  &\jdef
    {\jloc{a}{D}{P} \jcon \jrct{\jloc{a}{E}{Q}}{R}}
    {S}
  \\
  &\jdef
    {\jloc{a}{\jrct{\jrcv{x}{u}}{P}}{Q}
     \jcon
     \jloc{b}{\jrct{\jrcv{x}{v}}{R}}{S}
     }
    {T}
\end{align*}

The semantics of the new constructs are defined using a distributed version
of the RCHAM, called DRCHAM.
% TODO: explain DRCHAM

Then we add three new rules for the two new syntactical constructs:
% TODO: which \phi ?

\begin{align*}
  \drchamequ{str-loc}{(a \ \text{frozen)}}{
    \drchamcnf{\phi}{\jloc{a}{D}{P}}{}
  }{
    \drchamcnf{\phi}{}{}
    \drchampar
    \drchamcnf{\phi a}{D}{P}
  }
  \\
  \drchamred{comm}{(x \in \dv{J})}{
    \drchamcnf{\phi}{}{\jsnd{x}{\tilde{v}}}
    \drchampar
    \drchamcnf{}{\jrct{J}{P}}{}
  }{
    \drchamcnf{\phi}{}{}
    \drchampar
    \drchamcnf{}{\jrct{J}{P}}{\jsnd{x}{\tilde{v}}}
  }
  \\
  \drchamred{move}{}{
    \drchamcnf{\phi}{\jloc{a}{D}{P \jpar \jgo{b}{\kappa}}}{}
    \drchampar
    \drchamcnf{\psi b}{}{}
  }{
    \drchamcnf{\phi}{}{}
    \drchampar
    \drchamcnf{\psi b}{\jloc{a}{D}{P \jpar \jsnd{\kappa}{}}}{}
  }
\end{align*}

% TODO: explain new rules
% can dissolve new kind of definition as with str-def
% move emitted message to correct location
% migrate the current location

% conditions
% location tree
