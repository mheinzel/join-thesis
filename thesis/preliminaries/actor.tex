\section{The \ActorModel}

The notion of actors was first developed by Hewitt \cite{hewitt_universal_1973}
and later formalized by Agha \cite{agha_actors:_1986}
as a model of concurrent computation.
As described in \cite{agha_algebraic_2004},
a computational system in the \actormodel, called a \emph{configuration},
consists of a collection of concurrently executing actors
and a collection of messages in transit.
Each actor has a unique name (the \emph{uniqueness} property)
and a \emph{behavior},
and it communicates with other actors via asynchronous messages.
Actors are reactive in nature,
i.e. they execute only in response to messages receives.
An actor's behavior is deterministic in that its response to a message is
uniquely determined by the message contents.
Message delivery in the \actormodel is fair.
The delivery of a message can only be delayed
for a finite but unbounded amount of time.

An actor can perform three basic actions on receiving a message:
\begin{itemize}[nosep]
  \item create a finite number of actors with universally fresh names
  \item send a finite number of messages
  \item assume a new behavior
\end{itemize}

In addition to the already mentioned \emph{uniqueness},
there are two other properties we should note:
First, actors do not disappear after processing a message
(the \emph{persistence} property)
and second,
actors cannot be created with well known names or names received in a message
(the \emph{freshness} property).

The description of a configuration also defines an interface
between the configuration and its environment,
which constrains interactions between them.
An interface is a set of names, called the \emph{receptionist} set,
that contains the names of all actors in the configuration
that are visible to the environment.
The receptionist set may evolve during interactions,
as the messages that the configuration sends to the environment
may contain names of actors not currently in the receptionist set.

\subsection{The Actor Calculus \actorpicalc}

This definition of the \actormodel has many differences from
\joincalc and similar calculi.
While actors are uniquely identified,
processes in most calculi are anonymous.
The core idea that allows \actorpicalc to resolve this mismatch,
is to identify processes by the channel name they receive on.
To allow this, several restrictions are enforced by a type system.
The definition is directly taken from \cite{agha_algebraic_2004}.

\subsubsection{Syntax}

We assume an infinite set of channel names
$ \mathcal{N} = \{u, v, w, x, y, z, \ldots\} $.
The set of actor configurations is defined by the following grammar.

\begin{align*}
  P, Q
  \grmr \anullproc
  \altn \amrcv{x}{y} . P
  \altn \amsnd{x}{y}
  \altn \anew{x}{P}
  \altn P \apar Q
  \altn \acse{x}{\aalt{y_1}{P_1}, \dots, \aalt{y_n}{P_n}}
  \altn \ains{B}{\tilde{x}}{\tilde{y}}
\end{align*}

Order of precedence among the constructs is the order in which they are
listed.
The simplest terms are $\anullproc$, representing an empty configuration,
and output term $\amsnd{x}{y}$, which represents a configuration with a single message
targeted to $x$ containing a single name $y$.
We will consider polyadic communication towards the end of this chapter.

The input term $\arcv{x}{y} . P$ stands for a configuration with an actor
$x$ whose behavior is $(y) P$.
Its parameter $y$ constitutes the formal parameter list and binds all
free occurences of $y$ in $P$.
The actor $x$ can receive an arbitrary name $z$
and substitute it for $y$ to behave like $P \substitution{\subst{z}{y}}$.
%break?
A restricted process $\anew{x}{P}$ is the same as $P$,
except that $x$ is no longer a receptionist of $P$.
The composition $P_1 \apar P_2$ is a configuration containing
all the actors and messages in $P_1$ and $P_2$.
The configuration
$\acse{x}{\aalt{y_1}{P_1}, \ldots, \aalt{y_n}{P_n}}$
behaves like $P_i$ if $x = y_i$ for any $i$,
and like $\anullproc$ otherwise.
If more than one branch matches, one of them is chosen non-deterministically.
Finally, there are behavor instantiations
$\ains{B}{\tilde{u}}{\tilde{v}}$.
Any instantiated identifier $B$ must have a single defining equation of the form
$\adef{B}{\tilde{u}}{\tilde{v}}{\arcv{x_1}{z} . P}$,
where $\tilde{x}$ is a tuple of one or two distinct names
and $x_1$ denotes its first component.
Also, the tuples $\tilde{x}$ and $\tilde{y}$ together must contain
exactly the free names in $\arcv{x_1}{z} . P$.


\subsubsection{Type System}

Not all syntactically well-formed configurations are valid actor configurations.
For example, the term $\arcv{x}{u} . P \apar \arcv{x}{v} . Q$
contains two actors with name $x$,
which violates the uniqueness property.
Similarly, $\arcv{x}{u} . \arcv{u}{v} . P$ creates an actor
with a name $u$ that is received in a message,
violating the freshness property.
The actor properties can be enforced, however,
by imposing a type system that statically tracks the receptionists of each term.

Since requiring an actor to only ever receive on one single channel
is too restrictive, the persistence property is slightly relaxed.
Intuitively, an actor is allowed to temporarily assume a new name,
which requires the type system to track whether names
are regular or only temporarily assumed by an actor.

Before presenting the type system, a few notational conventions are in order.
For a tuple $\tilde{x}$ and $\tilde{y}$,
we denote the set of names occuring in $\tilde{x}$ by $\{\tilde{x}\}$
and the result of appending $y$ to $x$ by $y,x$.
Also, we use $\epsilon$ for the empty tuple
and assume a variable $\hat{z}$ to be either $\emptyset$ or $\{z\}$.

\pagebreak

We assume $\bot, * \notin \mathcal{N}$
and for $X \subset \mathcal{N}$
we define
$X^* = X \cup \{\bot, *\}$.
For $f : X \rightarrow X^*$
we define $f^* : X^* \rightarrow X^*$ as
$f^*(x) = f(x) $ for $ x \in X$ and
$f^*(\bot) = f^*(*) = \bot$.

A typing judgement is of the form
$\judgement{\rho}{f}{P}$,
where $\rho$ is the receptionist set of P,
and $f : \rho \rightarrow \rho^*$ is a temporary name mapping function
that relates actors in $P$ to the temporary names they have currently assumed.
Specifically:
\begin{itemize}[noitemsep]
  \item $f(x) = \bot$ means that $x$ is a regular actor name,
  \item $f(x) = y \notin \{\bot, *\}$ means that actor $y$
    has assumed the temporary name $x$, and
  \item $f(x) = *$ means $x$ is the temporary name
    of an actor with a private name (bound by restriction).
\end{itemize}

The function $f$ has the following properties,
which we call \emph{valid mapping} properties. For all $x, y \in \rho$,
\begin{itemize}[noitemsep]
  \item $f(x) \neq x$
  \item $f(x) = f(y) \notin \{\bot, *\}$ implies $x = y$,
    because an actor cannot assume
    more than one temporary name at the same time.
  \item $f^*(f(x)) = \bot$,
    because temporary names cannot themselves assume temporary names.
\end{itemize}

We define a few functions and relations
on the temporary name mapping functions.
Let $f_1: \rho_1 \to \rho_1^*$ and  $f_2: \rho_2 \to \rho_2^*$.

\begin{itemize}
  \item
    We define $f_1 \oplus f_2: \rho_1 \cup \rho_2 \to (\rho_1 \cup \rho_2)^*$ as
    \begin{equation*}
      (f_1 \oplus f_2)(x) =
      \begin{cases}
        f_1(x) &\text{if $x \in \rho_1$, and $f_1(x) \neq \bot$ or $x \notin \rho_2$}\\
        f_2(x) &\text{otherwise}
      \end{cases}
    \end{equation*}
    Note that $\oplus$ is associative.
  \item
    If $\rho\subset\rho_1$, we define $\restrict{f}{\rho}: \rho\to\rho^*$ as
    \begin{equation*}
      (\restrict{f}{\rho})(x) =
      \begin{cases}
        *    &\text{if $f(x) \in \rho_1 - \rho$}\\
        f(x) &\text{otherwise}
      \end{cases}
    \end{equation*}
  \item
    We say $f_1$ and $f_2$ are compatible if
    $f_1 \oplus f_2 = f_2 \oplus f_1 = f$
    has the \emph{valid mapping} properties.
  \item
    Also, for a tuple $\tilde{x}$ we define
    $\ch{\tilde{x}}: \{\tilde{x}\} \to \{\tilde{x}\}^*$
    as
    $\ch{\epsilon} = \{\}$
    and if $\operatorname{len}(\tilde{x}) = n$,
    $\ch{\tilde{x}}(x_i) = x_{i+1}$ for $1 \leq i < n$
    and
    $\ch{\tilde{x}}(x_n) = \bot$.
\end{itemize}

Intuitively,
$\ch{x}$ represents a configuration with a single receptionist $x$, and
$\ch{t,x}$ represents a receptionist $x$ that temporarily assumed the name $t$.
Note that $\ch{\tilde{x}}$ with $\operatorname{len}(\tilde{x}) > 2$
cannot have the \emph{valid mapping} properties.
\\

\begin{minipage}{0.9\textwidth}
  \axiomrule{nil}{}
    {\judgement{}{}{\anullproc}}
  \axiomrule{msg}{}
    {\judgement{}{}{\amsnd{x}{y}}}
  \infrule{act}
    {\begin{varwidth}{6cm}
      $ \rho - \{x\} = \tilde{z}, y \notin \rho,$ and \\
      $ f =
        \begin{cases}
          \ch{x, \tilde{z}} \text{ if } x \in \rho \\
          \ch{\epsilon, \tilde{z}} \text{ otherwise }
        \end{cases} $
     \end{varwidth}}
    {\judgement
      {\rho}
      {f}
      {P}}
    {\judgement
      {\{x\} \cup \tilde{z}}
      {\ch{x, \tilde{z}}}
      {\amrcv{x}{y} . P}}
  \infrule{case}{$f_i$ are mutually compatible}
    {\forall\ 1 \leq i \leq n:
      \judgement{\rho_i}{f_i}{P_i}}
    {\judgement
      {(\cup_i \rho_i)}
      {(\oplus_i f_i)}
      {\acse{x}
        {\aalt{y_1}{P_1}
        ,\ldots
        ,\aalt{y_n}{P_n}}}}
  \infruleII{comp}{$ \rho_1 \cap \rho_2 = \emptyset $}
    {\judgement{\rho_1}{f_1}{P_1}}
    {\judgement{\rho_2}{f_2}{P_2}}
    {\judgement{\rho_1 \cup \rho_2}{f_1 \oplus f_2}{P_1 \apar P_2}}
  \infrule{res}{}
    {\judgement{\rho}{f}{P}}
    {\judgement
      {\rho - \{x\}}
      {\restrict{f}{\rho - \{x\}}}
      {\anew{x}{P}}}
  \axiomrule{inst}{$\operatorname{len}(\tilde{x}) = 2$ implies $x_1 \neq x_2$}
    {\judgement
      {\{\tilde{x}\}}
      {\ch{\tilde{x}}}
      {\ains{B}{\tilde{x}}{\tilde{y}}}}
\end{minipage}
\\
\\
\\
In the \rulename{act} rule,
if $\hat{z} = \{z\}$, then actor $z$ has assumed temporary name $x$.
The conditions $y \notin \rho$ and $\rho - \{x\} = \hat{z}$
together guarantee the freshness property by ensuring that new actors
are created with fresh names.
Note that it is possible for $x$ to be a regular name
and disappear after receiving a message ($x \notin \rho$).
We interpret this as the actor $x$ having assumed a sink behavior,
consuming messages without reaction.
With this interpretation, the persistence property is not violated.

The compatibility check in \rulename{case} prevents errors such as
two actors in different branches assuming the same temporary name
or the same actor assuming different temporary names in different branches.
The \rulename{comp} rule is critical for the uniqueness property
by ensuring that two composed configurations do not contain actors
with the same name.
For the \rulename{inst} rule to be sound, for every definition
$\adef{B}{\tilde{x}}{\tilde{y}}{\arcv{x_1}{z} . P}$
the judgement
$\judgement{\{\tilde{x}\}}{\ch{x}}{\arcv{x_1}{z} . P}$
must be derivable.

\vfill

\subsubsection{Semantics}

The operational semantics of \actorpicalc is formally specified
using a labeled transition system (defined modulo alpha-equivalence).
The symmetric versions of \rulename{com}, \rulename{close}, \rulename{par}
are not shown.
Transition labels, which are also called actions, can be of five forms:
$\tau$ (silent action),
$\actoutfree{x}{y}$ (free output of message with target $x$ and content $y$),
$\actoutbound{x}{y}$ (bound output),
$\actinfree{x}{y}$ (free input),
and $\actinbound{x}{y}$ (bound input).
We let $\alpha$ range over all visible (non-$\tau$) actions $\mathcal{L}$,
and $\beta$ over all actions.

\begin{minipage}{0.9\textwidth}
\axiomrule{out}{}
  {\amsnd{x}{y} \apireduction{\actoutfree{x}{y}} \anullproc}
\axiomrule{inp}{}
  {\amrcv{x}{y} . P \apireduction{\actinfree{x}{z}} P\substitution{\subst{z}{y}}}
\infrule{binp}{$ y \notin \fn{P} $}
  {P \apireduction{\actinfree{x}{y}} P'}
  {P \apireduction{\actinbound{x}{y}} P'}
\infrule{res}{$ y \notin \names{\alpha} $}
  {P \apireduction{\alpha} P'}
  {\anew{y}{P} \apireduction{\alpha} \anew{y}{P'}}
\infrule{open}{$ x \neq y $}
  {P \apireduction{\actoutfree{x}{y}} P'}
  {\anew{y}{P} \apireduction{\actoutbound{x}{y}} P'}
\infrule{par}{$ \bn{\alpha} \cap \fn{P_2} = \emptyset $}
  {P_1 \apireduction{\alpha} P_1'}
  {P_1 \apar P_2 \apireduction{\alpha} P_1' \apar P_2}
\infruleII{com}{}
  {P_1 \apireduction{\actoutfree{x}{y}} P_1'}
  {P_2 \apireduction{\actinfree{x}{y}} P_2'}
  {P_1 \apar P_2 \apireduction{\actsilent} P_1' \apar P_2}
\infruleII{close}{$ y \notin \fn{P_2} $}
  {P_1 \apireduction{\actoutbound{x}{y}} P_1'}
  {P_2 \apireduction{\actinfree{x}{y}} P_2'}
  {P_1 \apar P_2 \apireduction{\actsilent} \anew{y}{P_1' \apar P_2}}
\axiomrule{brnch}{$ x = y_i $}
  {\acse{x}{\aalt{y_1}{P_1}, \ldots, \aalt{y_n}{P_n}} \apireduction{\actsilent} P_i}
\infrule{behv}{$ \adef{B}{\tilde{x}}{\tilde{y}}{\amrcv{x_1}{z} . P} $}
  {(\amrcv{x_1}{z} . P)\substitution{\subst{(\tilde{u},\tilde{v})}{(\tilde{x},\tilde{y})}}
    \apireduction{\alpha} P'}
  {\ains{B}{\tilde{x}}{\tilde{y}} \apireduction{\alpha} P'}
\end{minipage}

\vfill

\subsubsection{Structural Congruence}

The strucural congruence $\astructequ$ on processes
is the smallest possible congruence closed under the following rules
\cite{thati_theory_2003}.

\begin{align*}
  P \apar \anullproc
    &\astructequ P \\
  P \apar Q
    &\astructequ Q \apar P \\
  P \apar (Q \apar R)
    &\astructequ (P \apar Q) \apar R \\
  \anew{x}{\anullproc}
    &\astructequ \anullproc \\
  \anew{x}{\anew{y}{\anullproc}}
    &\astructequ \anew{y}{\anew{x}{\anullproc}} \\
  P \apar \anew{x}{Q}
    &\astructequ \anew*{x}{P \apar Q} \text{ if } x \notin \fn{P} \\
  \ains{B}{\tilde{u}}{\tilde{v}}
    &\astructequ (\arcv{x_1}{z} . P)\substitution{\subst{(\tilde{u},\tilde{v}}{(\tilde{x},\tilde{y}}}
    \text{ if }
    \begin{cases}
      \adef{B}{\tilde{x}}{\tilde{y}}{\arcv{x_1}{z} . P}\\
      \operatorname{len}(\tilde{u}) = \operatorname{len}(\tilde{x})\\
      \operatorname{len}(\tilde{v}) = \operatorname{len}(\tilde{y})
    \end{cases}
\end{align*}


\subsubsection{Polyadic Communication}

As shown in \cite{agha_algebraic_2004},
polyadic communication (messages containing multiple names)
can be encoded in monadic \actorpicalc.
For fresh $u, z$ we have

\begin{align*}
  \encPolyAP{\asnd{x}{y_1, \ldots, y_n}}
  &= \anew*{u}{\amsnd{x}{u} \apar \ains{S_1}{u}{y_1, \ldots, y_n}}
  \\
  \adef*{S_i}{u}{y_1, \ldots, y_n}
    {\amrcv{u}{z} . \parens{\amsnd{z}{y_i} \apar \ains{S_{i+1}}{u}{y_{i+1}, \ldots, y_n}}}
    && 1 \leq i < n
  \\
  \adef*{S_n}{u}{y_n}{\amrcv{u}{z} . \amsnd{z}{y_n}}
  \\
  \encPolyAP{\arcv{x}{y_1, \ldots, y_n} . P}
  &= \amrcv{x}{u} . \anew*{z}{\amsnd{u}{z} \apar \ains{R_1}{z,\hat{x}}{u,\tilde{a}}}
  \\
  \adef*{R_i}{z,\hat{x}}{u,\tilde{a}}
    {\amrcv{z}{y_i} . \parens{\amsnd{u}{z} \apar \ains{R_1}{z,\hat{x}}{u,\tilde{a}}}}
    && 1 \leq i < n
  \\
  \adef*{R_n}{z,\hat{x}}{u,\tilde{a}}
    {\amrcv{z}{y_n} . \parens{\amsnd{u}{z} \apar \encPolyAP{P}}}
\end{align*}

where
$\tilde{a} = \fn{\arcv{x}{y_1, \ldots, y_n} . P} - \{x\}$,
and
$\hat{x} = \{x\}$ if
$\judgement{\rho \cup \{x\}}{f}{\encPolyAP{P}}$ for some $\rho, f$
and $\hat{x} = \emptyset$ otherwise.

In the paper, it is argued that this encoding cannot be used
for polyadic reception on temporary names,
since they cannot assume temporary names themselves.
However, this is too restrictive.
The necessary condition for $\arcv{x}{y_1, \ldots, y_n} . P$ is
that $P$ is an actor with a regular name.
Therefore, $x$ can be temporary, as long is it not used in $P$ anymore.

Furthermore, we adapt the \textsc{msg} and \textsc{act} rules of the type system
to accomodate polyadic communication.
In \textsc{act}, we now not only need to prevent the creation of actors
with a single name received in a message, but for all of them.
Apart from that, the rules remain unchanged.

\infrule{act}
  {\begin{varwidth}{8cm}
    $\rho - \{x\} = \tilde{z}$, $y_i \notin \rho$ for all $1 \leq i \leq n$, and \\
    $ f =
      \begin{cases}
        \ch{x, \tilde{z}} \text{ if } x \in \rho \\
        \ch{\epsilon, \tilde{z}} \text{ otherwise }
      \end{cases} $
   \end{varwidth}}
  {\judgement
    {\rho}
    {f}
    {P}}
  {\judgement
    {\{x\} \cup \tilde{z}}
    {\ch{x, \tilde{z}}}
    {\arcv{x}{y_1, \ldots, y_n} . P}}
