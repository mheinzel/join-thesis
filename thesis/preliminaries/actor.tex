\section{The Actor Model}

\TODO{3cm}{definition, properties from \cite{agha_actors:_1986}}
% properties
%   uniqueness
%   freshness
%   persistence


\subsection{\ActorPiCalc}

Taken directly from \cite{agha_algebraic_2004}.

\subsubsection{Syntax}

We assume an infinite set of channel names
$ \mathcal{N} = \{u, v, w, x, y, z, \ldots\} $.
The set of actor configurations is defined by the following grammar.

% TODO: polyadic or not?
\begin{align*}
  P, Q
  \grmr \anullproc
  \altn \arcv{x}{y} . P
  \altn \asnd{x}{y}
  \altn \anew{x}{P}
  \altn P \apar Q
  \altn \acse{x}{\aalt{y_1}{P_1}, \dots, \aalt{y_n}{P_n}}
  \altn \ains{B}{\tilde{x}}{\tilde{y}}
\end{align*}

The order of precedence among the constructs is the order in which they are
listed.

\TODO{3cm}{explain constructs\\
  (every inst requires a bevahior definition of form...)}

\TODO{2cm}{polyadic communication on temporary name\\
  only possible for last reception on that name!}

\subsubsection{Type System}

\TODO{3cm}{substitution?, ch, compatibility, ...}

% TODO:
% arrangement of rules?
% axioms as rules okay?
\axiomrule{nil}{}
  {\judgement{}{}{\anullproc}}

\axiomrule{msg}{}
  {\judgement{}{}{\asnd{x}{y}}}

\infrule{act}
  {\begin{varwidth}{6cm}
    $ \rho - \{x\} = \tilde{z}, y \notin \rho,$ and \\
    $ f =
      \begin{cases}
        \ch{x, \tilde{z}} \text{ if } x \in \rho \\
        \ch{\epsilon, \tilde{z}} \text{ otherwise }
      \end{cases} $
   \end{varwidth}}
  {\judgement
    {\rho}
    {f}
    {P}}
  {\judgement
    {\{x\} \cup \tilde{z}}
    {\ch{x, \tilde{z}}}
    {\arcv{x}{y} . P}}

\infrule{case}{$f_i$ are mutually compatible}
  {\forall\ 1 \leq i \leq n:
    \judgement{\rho_i}{f_i}{P_i}}
  {\judgement
    {(\cup_i \rho_i)}
    {(\oplus_i f_i)}
    {\acse{x}
      {\aalt{y_1}{P_1}
      ,\ldots
      ,\aalt{y_n}{P_n}}}}

\infruleII{comp}{$ \rho_1 \cap \rho_2 = \emptyset $}
  {\judgement{\rho_1}{f_1}{P_1}}
  {\judgement{\rho_2}{f_2}{P_2}}
  {\judgement{\rho_1 \cup \rho_2}{f_1 \oplus f_2}{P_1 \apar P_2}}

\infrule{res}{}
  {\judgement{\rho}{f}{P}}
  {\judgement
    {\rho - \{x\}}
    {\restrict{f}{\rho - \{x\}}}
    {\anew{x}{P}}}

\axiomrule{inst}{$\operatorname{len}(\tilde{x}) = 2$ implies $x_1 \neq x_2$}
  {\judgement
    {\{\tilde{x}\}}
    {\ch{\tilde{x}}}
    {\ains{B}{\tilde{x}}{\tilde{y}}}}

\TODO{1cm}{behavior definitions need to be type correct}

\TODO{3cm}{how are actor properties enforced?}


\subsubsection{Semantics}

Labelled transition system module alpha-equivalence.
Symmetric versions of COM, CLOSE, PAR are not shown.
Transition labels are called actions and can be of five forms:
$\tau$ (silent action),
$\actoutfree{x}{y}$ (free output of message with target $x$ and content $y$),
$\actoutbound{x}{y}$ (bound output),
$\actinfree{x}{y}$ (free input),
and $\actinbound{x}{y}$ (bound input).
We let $\alpha$ range over all visible (non-$\tau$) actions.

\axiomrule{out}{}
  {\asnd{x}{y} \apireduction{\actoutfree{x}{y}} \anullproc}

\axiomrule{inp}{}
  {\arcv{x}{y} . P \apireduction{\actinfree{x}{z}} P\substitution{\subst{z}{y}}}

\infrule{binp}{$ y \notin \fn{P_2} $}
  {P \apireduction{\actinfree{x}{y}} P'}
  {P \apireduction{\actinbound{x}{y}} P'}

\infrule{res}{$ y \notin \names{\alpha} $}
  {P \apireduction{\alpha} P'}
  {\anew{y}{P} \apireduction{\alpha} \anew{y}{P'}}

\infrule{open}{$ x \neq y $}
  {P \apireduction{\actoutfree{x}{y}} P'}
  {\anew{y}{P} \apireduction{\actoutbound{x}{y}} P'}

\infrule{par}{$ \bn{\alpha} \cap \fn{P_2} = \emptyset $}
  {P_1 \apireduction{\alpha} P_1'}
  {P_1 \apar P_2 \apireduction{\alpha} P_1' \apar P_2}

\infruleII{com}{}
  {P_1 \apireduction{\actoutfree{x}{y}} P_1'}
  {P_2 \apireduction{\actinfree{x}{y}} P_2'}
  {P_1 \apar P_2 \apireduction{\actsilent} P_1' \apar P_2}

\infruleII{close}{$ y \notin \fn{P_2} $}
  {P_1 \apireduction{\actoutbound{x}{y}} P_1'}
  {P_2 \apireduction{\actinfree{x}{y}} P_2'}
  {P_1 \apar P_2 \apireduction{\actsilent} \anew{y}{P_1' \apar P_2}}

\axiomrule{brnch}{$ x = y_i $}
  {\acse{x}{\aalt{y_1}{P_1}, \ldots, \aalt{y_n}{P_n}} \apireduction{\actsilent} P_i}

\infrule{behv}{$ \adef{B}{\tilde{x}}{\tilde{y}}{\arcv{x_1}{z} . P} $}
  {(\arcv{x_1}{z} . P)\substitution{\subst{(\tilde{u},\tilde{v})}{(\tilde{x},\tilde{y})}}
    \apireduction{\alpha} P'}
  {\ains{B}{\tilde{x}}{\tilde{y}} \apireduction{\alpha} P'}

\TODO{1cm}{interpret rules?}
