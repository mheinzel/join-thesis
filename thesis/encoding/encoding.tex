\section{A Naive Approach}

When introduction \joincalc, Fournet also gave a simple encoding into
\asyncpicalc \cite{fournet_reflexive_1996}.
Considering the minor syntactical differences,
we can translate it to \actorpicalc.

Let us also define
$\tilde{z}_P$
for the remainder of this chapter as the tuple of names in
$\fn{P} \setminus {\{x, y, u, v\}}$.
This is necessary since $P$ might contain free names
that are now also free in the definition of $B$ and thus
have to be bound in the parameter list.

\begin{align*}
  \encMJtoAP{P \jpar Q}
  &= \encMJtoAP{P} \apar \encMJtoAP{Q}
  \\
  \encMJtoAP{\jsnd{x}{y}}
  &= \asnd{x}{y}
  \\
  \encMJtoAP{\jdef{\jrct{\jrcv{x}{u} \jpat \jrcv{y}{v}}{P}}{Q}}
  &= \anew{x}{\anew{y}{\parens{\ains{B}{x, y}{\tilde{z}} \apar \encMJtoAP{Q}}}}
  \\
  \text{with }
  \adef*{B}{x,y}{\tilde{z}}{\arcv{x}{u} . \arcv{y}{v} . \parens{\ains{B}{x, y}{\tilde{z}} \apar \encMJtoAP{P}}}
\end{align*}

The definition of behavior $B$ in this encoding, however, is not well-typed.
Assuming that $P$ has no recipients, we can see that
$\judgement{\{x,y\}}{\ch{x,y}}{\ains{B}{x,y}{\tilde{z}}} \apar \encMJtoAP{P}$,
but then \rulename{act} does not apply to
$\arcv{y}{v} . \parens{\ains{B}{x,y}{\tilde{z}} \apar \encMJtoAP{P}}$,
since $y$ is already in the recipients (but not in first position).

Swapping the input actions to arrive at
$\judgement{\{x,y\}}{\ch{x,y}}
  {\arcv{x}{u} . \parens{\ains{B}{x,y}{\tilde{z}} \apar \encMJtoAP{P}}}$
does not help, either, as we then have the same problem
when prefixing it by $\arcv{y}{v}$.
Indeed, this approach is fundamentally flawed.


\section{Forwarding}

The \emph{persistence} property of the \actormodel is only slightly relaxed
in \actorpicalc to temporarily allow assuming another name.
It is not possible for an actor to receive on two different channels
interchangeably.
This restriction can only be circumvented by sending
both $x$ and $y$ messages on the same channel,
for example using a forwarder.

\begin{align*}
  \adef*{B_{fw}}{x}{a}{\arcv{x}{i} . \parens{\asnd{a}{x,i} \apar \ains{B_{fw}}{x}{a}}}
\end{align*}

When $B_{fw}$ receives a payload $i$ on a channel $x$,
it forwards it to $a$, tagged with the name of the channel it was originally
sent on.
At the same time, it recursively assumes the same behavior again to
forward further messages.


\section{Storing Payloads}

When forwarding both $x$ and $y$ to the same actor, a problem arises.
Messages from $x$ and $y$ can occur in an arbitrary order.
If we, for example, receive multiple messages from $x$, we must either:
\begin{enumerate}[nosep]
  \item
    respawn all but one of them and wait for a $y$ message.
    This introduces divergence, since the respawned messages could be
    received and spawned indefinitely.
  \item
    store them all.
\end{enumerate}

The only available values in \actorpicalc are names,
but it is possible to build data structures from actors.
A list is an actor and can either be empty or constructed from a single
first element (\emph{head}) and a remaining list (\emph{tail}).
To find out which of these is the case, we send it a message with two
continuations (names) $k_n$ and $k_c$.

If it is empty, it will send an empty message on $k_n$:
\begin{align*}
  \adef*{B_{nil}}{l}{}{\arcv{l}{k_n,k_c} . \parens{\asnd{k_n}{} \apar \ains{B_{nil}}{l}{}}}
\end{align*}

If not, it sends head and tail on $k_c$:
\begin{align*}
  \adef*{B_{cons}}{l}{h,t}{\arcv{l}{k_n,k_c} . \parens{\asnd{k_c}{h,t} \apar \ains{B_{cons}}{l}{h,t}}}
\end{align*}


\section{Our Encoding}

Using forwarders and lists, we can define an actor which receives
forwarded messages from both channels $x$ and $y$.
If it receives more messages from one of these channels than from the other,
it stores the excess payloads in a list $l$ and remembers from which channel
these came using a flag $f$.
This is possible since there can always only be excessive messages on one of these channels;
otherwise they would have been joined already.
We have to make sure the free names in $P$ do not conflict with any of the names
$a$, $f$, $l$, $c$, $p$, $k_n$, $k_c$, $k_P$ and $k_a$
to give following behavior for the actor.

\begin{align*}
  \adef*{B_a^P}{a}{x,y,f,l,\tilde{z}_P}{\arcv{a}{c,p} . \anew{k_n}{\anew{k_c}{\anew{k_P}{\anew{k_a}{
    \\
    &\hspace{1cm}
    \begin{aligned}[t]
      \big(\ &\asnd{l}{k_n,k_c}
      \\
      \apar  &\arcv{k_n}{} . \anew{l'}{\parens{\ains{B_{cons}}{l'}{p, l} \apar \asnd{k_a}{c, l'}}}
      \\
      \apar  &\arcv{k_c}{h,t} .
        \begin{aligned}[t]
          \acse*{f}{\big(\ 
            &\aalt{x}{
              \begin{aligned}[t]
                \acse*{c}{\big(\ 
                  &\aalt{x}{\anew*{l'}{
                    \ains{B_{cons}}{l'}{p,l} \apar \asnd{k_a}{f,l'}
                  }}\\
                  &\aalt{y}{
                    \asnd{k_P}{h,p} \apar \asnd{k_a}{f,t}
                  }\ 
                \big)}
              \end{aligned}
            }\\
            &\aalt{y}{
              \begin{aligned}[t]
                \acse*{c}{\big(\ 
                  &\aalt{y}{\anew*{l'}{
                    \ains{B_{cons}}{l'}{p,l} \apar \asnd{k_a}{f,l'}
                  }}\\
                  &\aalt{x}{
                    \asnd{k_P}{p,h} \apar \asnd{k_a}{f,t}
                  }\ 
                \big)}
                \ \big)
              \end{aligned}
            }\ 
          }
        \end{aligned}
      \\
      \apar  &\arcv{k_P}{u,v} . \encMJtoAP{P}
      \\
      \apar  &\arcv{k_a}{f',l'} . \ains{B_a^P}{a}{x, y, f', l',\tilde{z}_P}
      \ \big)
    \end{aligned}
  }}}}} \\
\end{align*}

To encode a definition,
we now start one such joining actor (with initially no excess payloads),
a forwarder for each channel
and the encoded subprocess in which the definition is available.

\begin{align*}
  \encMJtoAP{P \jpar Q}
  &= \encMJtoAP{P} \apar \encMJtoAP{Q}
  \\
  \encMJtoAP{\jsnd{x}{y}}
  &= \asnd{x}{y}
  \\
  \encMJtoAP{\jdef{\jrct{\jrcv{x}{u} \jpat \jrcv{y}{v}}{P}}{Q}}
  &= \anew{a}{\anew{x}{\anew{y}{\anew{l}{
    \begin{aligned}[t]
      \big(\ &\ains{B_a^P}{a}{x, y, x, l, \tilde{z}_P} \\
      \apar  &\ains{B_{fw}}{x}{a} \apar \ains{B_{fw}}{y}{a} \\
      \apar  &\ains{B_{nil}}{l}{} \\
      \apar  &\encMJtoAP{Q}
      \ \big)
    \end{aligned}
  }}}}
\end{align*}

As the parallel operator is translated homomorphically
and each definition is translated into separate actors,
the encoding preserves distributability \cite{peters_distributability_2013}.


\subsection{Type Correctness}
\label{type_correctness}

\begin{theorem}
  \label{enctypecorrect}
  $\judgement{}{}{\encMJtoAP{J}}$ for all terms $J$ in \corejoincalc
\end{theorem}

\begin{proof}
  We prove the theorem by structural induction over the grammar of \corejoincalc.
  Let $x, y \in \mathcal{N}$ and $J$ any term in \corejoincalc.

  \begin{case}[Base Step]
    $J = \jsnd{x}{y}$.
    \\
    Then $\encMJtoAP{J} = \asnd{x}{y}$
    and thus $\judgement{}{}{\encMJtoAP{J}}$
    by \rulename{msg}.
  \end{case}

  \begin{case}
    $J = P \jpar Q$
    \\
    Then we know by inductive hypothesis that
    $\judgement{}{}{\encMJtoAP{P}}$ and $\judgement{}{}{\encMJtoAP{Q}}$.
    Since $\encMJtoAP{P \jpar Q} = \encMJtoAP{P} \apar \encMJtoAP{Q}$,
    we can infer $\judgement{}{}{\encMJtoAP{J}}$
    by \rulename{comp}.
  \end{case}

  \begin{case}
    $J = \jdef{\jrct{\jrcv{x}{u} \jpat \jrcv{y}{v}}{P}}{Q}$
    \\
    Then we know by inductive hypothesis that
    $\judgement{}{}{\encMJtoAP{P}}$ and $\judgement{}{}{\encMJtoAP{Q}}$.

    First, all used definitions have to be well-typed.
    \begin{align*}
      \adef*{B_{fw}}{x}{a}{\arcv{x}{i} . \parens{\asnd{a}{x,i} \apar \ains{B_{fw}}{x}{a}}}
      &\judgement*{\{x\}}{\ch{x}}{\arcv{x}{i} . \parens{\asnd{a}{x,i} \apar \ains{B_{fw}}{x}{a}}}
      \\
      \adef*{B_{nil}}{l}{}{\arcv{l}{k_n,k_c} . \parens{\asnd{k_n}{} \apar \ains{B_{nil}}{l}{}}}
      &\judgement*{\{l\}}{\ch{l}}{\arcv{l}{k_n,k_c} . \parens{\asnd{k_n}{} \apar \ains{B_{nil}}{l}{}}}
      \\
      \adef*{B_{cons}}{l}{h,t}{\arcv{l}{k_n,k_c} . \parens{\asnd{k_c}{h,t} \apar \ains{B_{cons}}{l}{h,t}}}
      &\judgement*{\{l\}}{\ch{l}}{\arcv{l}{k_n,k_c} . \parens{\asnd{k_c}{h,t} \apar \ains{B_{cons}}{l}{h,t}}}
    \end{align*}
    \begin{align*}
      \adef*{B_a^P}{a}{x,y,f,l,\tilde{z}_P}{\arcv{a}{c,p} . \anew{k_n}{\anew{k_c}{\anew{k_P}{\anew{k_a}{
        \\
        &\hspace{1cm}
        R \left\lbrace
        \begin{aligned}[c]
          \big(\ &\asnd{l}{k_n,k_c}
          \\
          \apar  &\arcv{k_n}{} . \anew{l'}{\parens{\ains{B_{cons}}{l'}{p, l} \apar \asnd{k_a}{c, l'}}}
          \\
          \apar  &\arcv{k_c}{h,t} .
            \begin{aligned}[t]
              \acse*{f}{\big(\ 
                &\aalt{x}{
                  \begin{aligned}[t]
                    \acse*{c}{\big(\ 
                      &\aalt{x}{\anew*{l'}{
                        \ains{B_{cons}}{l'}{p,l} \apar \asnd{k_a}{f,l'}
                      }}\\
                      &\aalt{y}{
                        \asnd{k_P}{h,p} \apar \asnd{k_a}{f,t}
                      }\ 
                    \big)}
                  \end{aligned}
                }\\
                &\aalt{y}{
                  \begin{aligned}[t]
                    \acse*{c}{\big(\ 
                      &\aalt{y}{\anew*{l'}{
                        \ains{B_{cons}}{l'}{p,l} \apar \asnd{k_a}{f,l'}
                      }}\\
                      &\aalt{x}{
                        \asnd{k_P}{p,h} \apar \asnd{k_a}{f,t}
                      }\ 
                    \big)}
                    \ \big)
                  \end{aligned}
                }\ 
              }
            \end{aligned}
          \\
          \apar  &\arcv{k_P}{u,v} . \encMJtoAP{P}
          \\
          \apar  &\arcv{k_a}{f',l'} . \ains{B_a^P}{a}{x, y, f', l',\tilde{z}_P}
          \ \big)
        \end{aligned}
        \right.
      }}}}}
    \end{align*}

    Since none of the branches in the nested case-term have any recipients
    and the last line has type $\ch{k_a,a}$,
    we can derive:
    \begin{align*}
      \judgement*{\{k_n,k_c,k_P,k_a,a\}}{g}{R}
      \text{ \ with }
      g(k_n) = g(k_c) = g(k_P) = g(a) = \bot, g(k_a) = a
      \\
      \judgement*{\{a\}}{\ch{a}}{\anew{k_n}{\anew{k_c}{\anew{k_P}{\anew{k_a}{R}}}}}
      \\
      \judgement*{\{a\}}{\ch{a}}{\arcv{a}{c,p} . \anew{k_n}{\anew{k_c}{\anew{k_P}{\anew{k_a}{R}}}}}
    \end{align*}

    Now that we know that all definitions are correct,
    we can infer the type of $\encMJtoAP{J}$:

    \begin{align*}
      \encMJtoAP{J} = \anew{a}{\anew{x}{\anew{y}{\anew{l}{
        \underbrace{
          \left( \ains{B_a^P}{a}{x, y, x, l, \tilde{z}_P}
          \apar  \ains{B_{fw}}{x}{a} \apar \ains{B_{fw}}{y}{a}
          \apar  \ains{B_{nil}}{l}{}
          \apar  \encMJtoAP{Q}
          \right)
        }_{\textstyle S}
      }}}}
    \end{align*}
    Then
    $\judgement{\{a,x,y,l\}}{f}{S}$ with $f(a) = f(x) = f(y) = f(l) = \bot$
    and
    $\judgement{}{}{\encMJtoAP{J}}$. \hfill$\square$
 \end{case}
\end{proof}


\subsection{Semantic Correctness}

In \joincalc, we can receive in a join-pattern using following reduction:

\begin{align*}
  \jdef{\jrct{\jrcv{x}{u} \jpat \jrcv{y}{v}}{P}}{\jsnd{x}{b} \jpar \jsnd{y}{c} \jpar Q}
  \joinreduction
  \jdef{\jrct{\jrcv{x}{u} \jpat \jrcv{y}{v}}{P}}{P\substitution{\subst{b}{u},\subst{c}{v}} \jpar Q}
\end{align*}

To see that the proposed encoding allows the same behavior,
we can look at the reductions possible for the corresponding \actorpicalc term.
We will make use of the structural congruence to simplify the term between reductions
and focus on the interesting aspects.

\begin{align*}
  \encMJtoAP{\jdef{\jrct{\jrcv{x}{u} \jpat \jrcv{y}{v}}{P}}{\jsnd{x}{b} \jpar \jsnd{y}{c} \jpar Q}}
  \astructequ\ &\anew{a}{\anew{x}{\anew{y}{\anew{l}{ \\
      &\hspace{0.5cm}
      \begin{aligned}[t]
        \big(\ &\ains{B_{fw}}{x}{a}
        \apar   \ains{B_{fw}}{y}{a} \\
        \apar  &\asnd{x}{b} \apar \asnd{y}{c} \apar \encMJtoAP{Q} \\
        \apar  &\ains{B_{nil}}{l}{}
        \apar   \ains{B_a^P}{a}{x, y, x, l, \tilde{z}_P}
        \ \big)
      \end{aligned}
    }}}}
\end{align*}

In this configuration,
the messages $\asnd{x}{b}$ and $\asnd{y}{c}$ can be received by the forwarders
$\ains{B_{fw}}{x}{a}$ and $\ains{B_{fw}}{x}{a}$ respectively,
emitting the forwarded messages $\asnd{a}{x,b}$ and $\asnd{a}{y,c}$.

\begin{align*}
  &\anew{a}{\anew{x}{\anew{y}{\anew{l}{ \\
    &\hspace{0.5cm}
    \begin{aligned}[t]
      \big(\ &\asnd{a}{x,b}
      \apar   \ains{B_{fw}}{x}{a}
      \apar   \asnd{a}{y,c}
      \apar   \ains{B_{fw}}{y}{a}
      \apar   \encMJtoAP{Q}
      \apar   \ains{B_{nil}}{l}{} \\
      \apar  &\ains{B_a^P}{a}{x, y, x, l, \tilde{z}_P}
      \ \big)
    \end{aligned}
  }}}}
\end{align*}

Any of these can now be received by $B_a$. We pick $\asnd{a}{x,b}$ first,
but the other case works similarly.
To receive, we have to inline $B_a$'s definition, binding all its parameters.
Also, we pull all introduced name scopes to the outside.

\begin{align*}
  &\anew{a}{\anew{x}{\anew{y}{\anew{l}{\anew{k_n}{\anew{k_c}{\anew{k_P}{\anew{k_a}{ \\
    &\hspace{0.5cm}
    \begin{aligned}[t]
      \big(\ &\ains{B_{fw}}{x}{a}
      \apar   \asnd{a}{y,c}
      \apar   \ains{B_{fw}}{y}{a}
      \apar   \encMJtoAP{Q}
      \apar   \ains{B_{nil}}{l}{}
      \\
      \apar  &\asnd{l}{k_n,k_c}
      \\
      \apar  &\arcv{k_n}{} . \anew{l'}{\parens{\ains{B_{cons}}{l'}{b, l} \apar \asnd{k_a}{x, l'}}}
      \\
      \apar  &\arcv{k_c}{h,t} .
        \begin{aligned}[t]
          \acse*{f}{\big(\ 
            &\aalt{x}{
              \begin{aligned}[t]
                \acse*{x}{\big(\ 
                  &\aalt{x}{\anew*{l'}{
                    \ains{B_{cons}}{l'}{b,l} \apar \asnd{k_a}{f,l'}
                  }}\\
                  &\aalt{y}{
                    \asnd{k_P}{h,b} \apar \asnd{k_a}{f,t}
                  }\ 
                \big)}
              \end{aligned}
            }\\
            &\aalt{y}{
              \begin{aligned}[t]
                \acse*{x}{\big(\ 
                  &\aalt{y}{\anew*{l'}{
                    \ains{B_{cons}}{l'}{b,l} \apar \asnd{k_a}{f,l'}
                  }}\\
                  &\aalt{x}{
                    \asnd{k_P}{b,h} \apar \asnd{k_a}{f,t}
                  }\ 
                \big)}
                \ \big)
              \end{aligned}
            }\ 
          }
        \end{aligned}
      \\
      \apar  &\arcv{k_P}{u,v} . \encMJtoAP{P}
      \\
      \apar  &\arcv{k_a}{f',l'} . \ains{B_a^P}{a}{x, y, f', l',\tilde{z}_P}
      \ \big)
    \end{aligned}
  }}}}}}}}
\end{align*}

Now, $\asnd{l}{k_n,k_c}$ is received by $\ains{B_{nil}}{l}{}$,
which (since it is empty), emits $\jsnd{k_n}{}$.
We use this opportunity to clean up the term a bit.
After this step, the only occurences of $k_c$ and $k_P$ are where they are
received or in a process blocked by such a reception. Using the
structural congruence, we can limit the scope of these names and arrive at
$\anew{k_c}{\anew*{k_P}{\arcv{k_c}{h,t} . P_{k_c} \apar \arcv{k_P}{u,v} . P_{k_P}}}$
for the corresponding $P_{k_c}$ and $P_{k_P}$.

This process cannot interact with its context or be reduced any further.
To do so, a message on $k_c$ or $k_P$ would be necessary, but it can never occur.
We denote this term as $R_{garbage}$.

\begin{align*}
  &\anew{a}{\anew{x}{\anew{y}{\anew{l}{\anew{k_n}{\anew{k_a}{ \\
    &\hspace{0.5cm}
    \begin{aligned}[t]
      \big(\ &\ains{B_{fw}}{x}{a}
      \apar   \asnd{a}{y,c}
      \apar   \ains{B_{fw}}{y}{a}
      \apar   \encMJtoAP{Q}
      \apar   \asnd{k_n}{}
      \apar   \ains{B_{nil}}{l}{}
      \\
      \apar  &\arcv{k_n}{} . \anew{l'}{\parens{\ains{B_{cons}}{l'}{b, l} \apar \asnd{k_a}{x, l'}}}
      \\
      \apar  &\arcv{k_a}{f',l'} . \ains{B_a^P}{a}{x, y, f', l',\tilde{z}_P}
      \\
      \apar  &R_{garbage}
      \ \big)
    \end{aligned}
  }}}}}}
\end{align*}

We then receive $\asnd{k_n}{}$ in the third line
and pull outwards the scope of $l'$.

\begin{align*}
  &\anew{a}{\anew{x}{\anew{y}{\anew{l}{\anew{k_a}{\anew{l'}{ \\
    &\hspace{0.5cm}
    \begin{aligned}[t]
      \big(\ &\ains{B_{fw}}{x}{a}
      \apar   \asnd{a}{y,c}
      \apar   \ains{B_{fw}}{y}{a}
      \apar   \encMJtoAP{Q}
      \apar   \ains{B_{nil}}{l}{}
      \\
      \apar  &\ains{B_{cons}}{l'}{b, l}
      \apar   \asnd{k_a}{x, l'}
      \\
      \apar  &\arcv{k_a}{f',l'} . \ains{B_a^P}{a}{x, y, f', l',\tilde{z}_P}
      \\
      \apar  &R_{garbage}
      \ \big)
    \end{aligned}
  }}}}}}
\end{align*}

The continuation $k_a$ is received next. Afterwards, it does not occur at all,
which allows us to completely remove its scope using the structural congruence:

$ \anew{k_a}{S}
  \astructequ \anew*{k_a}{S \apar \anullproc}
  \astructequ S \apar \anew{k_a}{\anullproc}
  \astructequ S \apar \anullproc
  \astructequ S
  \text{ \ (for any $S$ without $k_a$)} $

\begin{align*}
  &\anew{a}{\anew{x}{\anew{y}{\anew{l}{\anew{l'}{ \\
    &\hspace{0.5cm}
    \begin{aligned}[t]
      \big(\ &\ains{B_{fw}}{x}{a}
      \apar   \asnd{a}{y,c}
      \apar   \ains{B_{fw}}{y}{a}
      \apar   \encMJtoAP{Q}
      \apar   \ains{B_{nil}}{l}{}
      \apar   \ains{B_{cons}}{l'}{b, l}
      \\
      \apar  &\ains{B_a^P}{a}{x, y, x, l',\tilde{z}_P}
      \\
      \apar  &R_{garbage}
      \ \big)
    \end{aligned}
  }}}}}
\end{align*}

At this point, it is important to note that we are almost in the same situation
as in the beginning,
but $B_a$ successfully received its first message from $x$ and stored
the payload $b$ in list $l'$.
If there was another message $\asnd{x}{d}$, it could be added to the list as well,
resulting in
$\anew{l''}{\anew{l'}{\anew*{l}{
  \cdots \apar \ains{B_{cons}}{l''}{d,l'} \apar \ains{B_{cons}}{l'}{b,l} \apar \ains{B_{nil}}{l}{}
}}}$.

The process can now receive the complimentary message $\asnd{a}{y,c}$.
In its body, we alpha-rename the occurences of $l'$ to $l''$ to avoid conflicts.

\begin{align*}
  &\anew{a}{\anew{x}{\anew{y}{\anew{l}{\anew{l'}{\anew{k_n}{\anew{k_c}{\anew{k_P}{\anew{k_a}{ \\
    &\hspace{0.5cm}
    \begin{aligned}[t]
      \big(\ &\ains{B_{fw}}{x}{a}
      \apar   \ains{B_{fw}}{y}{a}
      \apar   \encMJtoAP{Q}
      \apar   \ains{B_{nil}}{l}{}
      \apar   \ains{B_{cons}}{l'}{b, l}
      \\
      \apar  &\asnd{l'}{k_n,k_c}
      \\
      \apar  &\arcv{k_n}{} . \anew{l''}{\parens{\ains{B_{cons}}{l''}{c, l'} \apar \asnd{k_a}{y, l''}}}
      \\
      \apar  &\arcv{k_c}{h,t} .
        \begin{aligned}[t]
          \acse*{x}{\big(\ 
            &\aalt{x}{
              \begin{aligned}[t]
                \acse*{y}{\big(\ 
                  &\aalt{x}{\anew*{l''}{
                    \ains{B_{cons}}{l''}{c,l'} \apar \asnd{k_a}{x,l''}
                  }}\\
                  &\aalt{y}{
                    \asnd{k_P}{h,c} \apar \asnd{k_a}{x,t}
                  }\ 
                \big)}
              \end{aligned}
            }\\
            &\aalt{y}{
              \begin{aligned}[t]
                \acse*{y}{\big(\ 
                  &\aalt{y}{\anew*{l''}{
                    \ains{B_{cons}}{l''}{c,l'} \apar \asnd{k_a}{x,l''}
                  }}\\
                  &\aalt{x}{
                    \asnd{k_P}{c,h} \apar \asnd{k_a}{x,t}
                  }\ 
                \big)}
                \ \big)
              \end{aligned}
            }\ 
          }
        \end{aligned}
      \\
      \apar  &\arcv{k_P}{u,v} . \encMJtoAP{P}
      \\
      \apar  &\arcv{k_a}{f',l''} . \ains{B_a^P}{a}{x, y, f', l'',\tilde{z}_P}
      \\
      \apar  &R_{garbage}
      \ \big)
    \end{aligned}
  }}}}}}}}}
\end{align*}

We now perform multiple reductions at once:
First, $\ains{B_{cons}}{l'}{b,l}$ receives $\asnd{l'}{k_n,k_c}$.
Then (because it is not empty) it sends $\asnd{k_c}{b,l}$,
which is in turn received.

Also, we remove the scope of $k_c$ and add the continuation $k_n$ to
the existing garbage (resulting in $R_{garbage}'$).

\begin{align*}
  &\anew{a}{\anew{x}{\anew{y}{\anew{l}{\anew{l'}{\anew{k_P}{\anew{k_a}{ \\
    &\hspace{0.5cm}
    \begin{aligned}[t]
      \big(\ &\ains{B_{fw}}{x}{a}
      \apar   \ains{B_{fw}}{y}{a}
      \apar   \encMJtoAP{Q}
      \apar   \ains{B_{nil}}{l}{}
      \apar   \ains{B_{cons}}{l'}{b, l}
      \\
      \apar
        &\begin{aligned}[t]
          \acse*{x}{\big(\ 
            &\aalt{x}{
              \begin{aligned}[t]
                \acse*{y}{\big(\ 
                  &\aalt{x}{\anew*{l''}{
                    \ains{B_{cons}}{l''}{c,l'} \apar \asnd{k_a}{x,l''}
                  }}\\
                  &\aalt{y}{
                    \asnd{k_P}{b,c} \apar \asnd{k_a}{x,l}
                  }\ 
                \big)}
              \end{aligned}
            }\\
            &\aalt{y}{
              \begin{aligned}[t]
                \acse*{y}{\big(\ 
                  &\aalt{y}{\anew*{l''}{
                    \ains{B_{cons}}{l''}{c,l'} \apar \asnd{k_a}{x,l''}
                  }}\\
                  &\aalt{x}{
                    \asnd{k_P}{c,b} \apar \asnd{k_a}{x,l}
                  }\ 
                \big)}
                \ \big)
              \end{aligned}
            }\ 
          }
        \end{aligned}
      \\
      \apar  &\arcv{k_P}{u,v} . \encMJtoAP{P}
      \\
      \apar  &\arcv{k_a}{f',l''} . \ains{B_a^P}{a}{x, y, f', l'',\tilde{z}_P}
      \\
      \apar  &R_{garbage}'
      \ \big)
    \end{aligned}
  }}}}}}}
\end{align*}

At this point it becomes apparent which of the cases applies.

\begin{align*}
  &\anew{a}{\anew{x}{\anew{y}{\anew{l}{\anew{l'}{\anew{k_P}{\anew{k_a}{ \\
    &\hspace{0.5cm}
    \begin{aligned}[t]
      \big(\ &\ains{B_{fw}}{x}{a}
      \apar   \ains{B_{fw}}{y}{a}
      \apar   \encMJtoAP{Q}
      \apar   \ains{B_{nil}}{l}{}
      \apar   \ains{B_{cons}}{l'}{b, l}
      \\
      \apar  &\asnd{k_P}{b,c}
      \apar   \asnd{k_a}{x,l}
      \\
      \apar  &\arcv{k_P}{u,v} . \encMJtoAP{P}
      \\
      \apar  &\arcv{k_a}{f',l''} . \ains{B_a^P}{a}{x, y, f', l'',\tilde{z}_P}
      \\
      \apar  &R_{garbage}'
      \ \big)
    \end{aligned}
  }}}}}}}
\end{align*}

The continuation $k_P$ passes the correct parameters to $\encMJtoAP{P}$.

\begin{align*}
  &\anew{a}{\anew{x}{\anew{y}{\anew{l}{\anew{l'}{\anew{k_a}{ \\
    &\hspace{0.5cm}
    \begin{aligned}[t]
      \big(\ &\ains{B_{fw}}{x}{a}
      \apar   \ains{B_{fw}}{y}{a}
      \apar   \encMJtoAP{Q}
      \apar   \ains{B_{nil}}{l}{}
      \apar   \ains{B_{cons}}{l'}{b, l}
      \\
      \apar  &\asnd{k_a}{x,l}
      \\
      \apar  &\encMJtoAP{P}\substitution{\subst{b}{u},\subst{c}{v}}
      \\
      \apar  &\arcv{k_a}{f',l''} . \ains{B_a^P}{a}{x, y, f', l'',\tilde{z}_P}
      \\
      \apar  &R_{garbage}'
      \ \big)
    \end{aligned}
  }}}}}}
\end{align*}

As a last step, the continuation $k_a$ prepares $B_a$ for receiving further
messages, with an empty list of payloads again.

\begin{align*}
  &\anew{a}{\anew{x}{\anew{y}{\anew{l}{\anew{l'}{ \\
    &\hspace{0.5cm}
    \begin{aligned}[t]
      \big(\ &\ains{B_{fw}}{x}{a}
      \apar   \ains{B_{fw}}{y}{a}
      \apar   \encMJtoAP{Q}
      \apar   \ains{B_{nil}}{l}{}
      \apar   \ains{B_{cons}}{l'}{b, l}
      \\
      \apar  &\encMJtoAP{P}\substitution{\subst{b}{u},\subst{c}{v}}
      \\
      \apar  &\ains{B_a^P}{a}{x, y, x, l,\tilde{z}_P}
      \\
      \apar  &R_{garbage}'
      \ \big)
    \end{aligned}
  }}}}}
\end{align*}

We can make two observations:
The substitution can be moved into the encoding, since it is name invariant.
Also, we can add $\anew{l'}{\ains{B_{cons}}{l'}{b,l}}$ to the garbage.

\begin{align*}
  &\anew{a}{\anew{x}{\anew{y}{\anew{l}{ \\
    &\hspace{0.5cm}
    \begin{aligned}[t]
      \big(\ &\ains{B_{fw}}{x}{a}
      \apar   \ains{B_{fw}}{y}{a}
      \apar   \encMJtoAP{Q}
      \apar   \ains{B_{nil}}{l}{}
      \\
      \apar  &\encMJtoAP{P\substitution{\subst{b}{u},\subst{c}{v}}}
      \\
      \apar  &\ains{B_a^P}{a}{x, y, x, l,\tilde{z}_P}
      \\
      \apar  &R_{garbage}''
      \ \big)
    \end{aligned}
  }}}}
\end{align*}

Comparing this configuration to the encoding of the reduction result
in \joincalc,
we see that it is structurally equivalent, with the exception of $R_{garbage}''$.
As the garbage only consists of terms that cannot interact anymore,
we conclude that both configurations have the same behavior.

\begin{align*}
  \encMJtoAP{\jdef{\jrct{\jrcv{x}{u} \jpat \jrcv{y}{v}}{P}}{P\substitution{\subst{b}{u},\subst{c}{v}} \jpar Q}}
  =\ &\anew{a}{\anew{x}{\anew{y}{\anew{l}{ \\
      &\hspace{0.5cm}
      \begin{aligned}[t]
        \big(\ &\ains{B_{fw}}{x}{a}
        \apar   \ains{B_{fw}}{y}{a} \\
        \apar  &\encMJtoAP{P\substitution{\subst{b}{u},\subst{c}{v}}} \apar \encMJtoAP{Q} \\
        \apar  &\ains{B_{nil}}{l}{}
        \apar   \ains{B_a^P}{a}{x, y, x, l, \tilde{z}_P}
        \ \big)
      \end{aligned}
    }}}}
\end{align*}
