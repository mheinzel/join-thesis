\section{Distributed Implementation}

We will now extend the implementation by adding the possibility to
create, refer to and migrate locations as in \distjoincalc.


\subsection{Migrating Actors}
\label{ch_migration}

Running Erlang processes cannot be migrated.
However, functions can be serialized, sent somewhere else and spawned there.
Since our provided primitives are the only places where actors are created,
we can equip them with the ability to stop their execution
and return a function capturing their behavior together with their
current state (which consists only of the values of their parameters).
For example, see the extended definition of the \code{forward} behavior
(figure \ref{distributed/wrap_up_forward}).
The returned data structure can then be sent to another node
and re-spawned there.

\thesislisting{distributed/wrap_up_forward}{the extended forward behavior}

This creates a new process with a new \PID,
but it should still be reachable by its name.
This requires us to use our own globally unique names as actor addresses.
Fortunately, the Erlang ecosystem already provides global process registries
such as \emph{global}, which we just wrap in the \code{join_reg} module.

Another problem we have to consider is that a migrating actor,
when it handles the \code{wrap_up} message and halts execution,
might still have messages in its inbox.
Worse, even after stopping execution,
nodes that are not yet aware of the migration
could try to send a message to the old \PID.
% TODO: and we don't want to globally lock
% TODO: problems with gproc?
To prevent the loss of messages, migrating actors will immediately unregister
from their name, wait until it becomes registered somewhere else
and stay alive for a limited time to forward any remaining messages
to the new process.


\subsection{Locations}

\TODO{3cm}{representation in code, wrapping up}

\TODO{2cm}{physical root locations (unmovable)}

\TODO{3cm}{CODE FOR NEW DEF}

\TODO{2cm}{passing locations explicitly}


\subsection{Connecting Nodes}

\TODO{3cm}{need to find other nodes\\
  but no location server\\
  solution is to register names globally (but without explicit nameserver)}

\TODO{3cm}{distributed examples\\
  (explain how to start nodes and connect them)}
