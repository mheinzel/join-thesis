Based on the presented encoding,
we will now create the primitives of \corejoincalc
in a language that allows us to create actor configurations.
For this task, we chose \emph{Erlang}.
In contrast to other languages offering actor frameworks,
Erlang is based on the actor model from its core principles.
It allows us to spawn actors, referred to as \emph{Erlang processes},
and asynchronically send messages to them by their \emph{process id} (\PID).
In contrast, when we use the word \emph{actor},
we refer to the concept from \actorpicalc:
A \actorname with a behavior and values bound to its parameters.

While \PID{}s are globally unique and can be used to send messages to
processes located on another node,
they are not flexible enough for our use case
(the reasons are explained in chapter \ref{ch_migration}).
Thus, we create our own, globally unique \actorname{}s to refer to actors.
This is similar to the construct $\anew{x}{P}$ in \actorpicalc.
To be able to send messages using such a \actorname,
we use our own \code{send} function and require every actor to register itself
(\code{register_self}),
so we can maintain a registry mapping from \actorname{}s to \PID{}s.
The implementation of these functions will be explored later.


\section{Basic Translation}
\label{implementation_basic}

Terms of \actorpicalc can be translated into Erlang in a straightforward way.
See for example the equivalent of the behavior definition $B_{fw}$
(figure \ref{simple/forward}).
The main difference is that the emission $\asnd{a}{x,i}$
and recursive instantiation $\ains{B_{fw}}{x}{a}$
are composed in parallel in \actorpicalc,
but sequentially in Erlang.
However, the message will be sent asynchronously,
so its transmission and reception happen concurrently to the instantiation
in the next line.
Generally, we will spawn multiple Erlang processes to
mirror parallel composition in \actorpicalc terms.

\thesislisting{simple/forward}{behavior definition of a forwarder}

The translation of the behavior $B_a^P$ is a bit more complex
(figure \ref{simple/a}).
To obtain it, we must first pass in the process $P$,
which is not a formal behavior parameter.
The behavior itself then receives the expected parameters.
Since Erlang provides built-in lists and case-expressions,
we do not need to use church encoding and continuation passing anymore.

\thesislisting{simple/a}{behavor definition of $B_a^P$}

While this is the simplest and conceptually closest translation,
the real code uses a queue in place of the list of payloads.
This ensures fairness, since messages that reach the actor first
will also be joined first.

After defining the necessary behaviors,
we can now translate the encoding itself.
First we have to decide what the primitives of \joincalc should look like
in Erlang.
Sending a message is just a function call and parallel composition
is not necessary when all provided actions are asynchronous.
A join definition
$\jdef{\jrct{\jrcv{x}{u} \jpat \jrcv{y}{v}}{P}}{Q}$
can be modelled using a function that receives two processes $P$ and $Q$,
where both should have access to the channel names $x$ and $y$,
and $P$ to $u$ and $v$ as well.
To model this, we pass a function that receives the shared names and
returns two functions (the first one taking the additional names).
We can use it as shown in figure \ref{simple/def_usage}.
The syntax is noisier than in the calculus, but the structure is recognizable.

\thesislisting{simple/def_usage}{creating a join definition}

The \code{def} function (figure \ref{simple/def_encoding})
is again in close correspondence with the formal encoding.
We create new \actorname{}s and spawn all necessary actors
using the \code{spawn_actor} function (figure \ref{simple/spawn_actor}),
which also registers them in our name registry.
Note that $Q$ is not registered, since it is not an actor.
As shown in section \ref{type_correctness},
the encoding of any process in \joincalc has no recipients.

\thesislisting{simple/def_encoding}{encoding of a join definition}

\thesislisting{simple/spawn_actor}{registering and spawning an actor}

\begin{example}
  \label{example_print_erlang}
  Terms such as
  $ \ \jdef
      {\jrct{\jrcv{x}{u} \jpat \jrcv{y}{v}}{\texttt{print($u$,$v$)}}}
      {\jsnd{x}{\texttt{"a"}}
        \jpar \jsnd{y}{1}
        \jpar \jsnd{y}{2}} \ $
  can now be expressed in Erlang
  (figure \ref{simple_examples/print}).

  \thesislisting{simple_examples/print}
  {join-definition (example \ref{example_print_erlang})}
\end{example}

\begin{example}
  Also, we can create a function for single channel join-patterns using the
  existing primitives
  (figure \ref{simple_examples/continuation}).
  Such a channel can then be used as a continuation.

  \thesislisting{simple_examples/continuation}
  {a function for join-patterns consisting of only one channel}
\end{example}
