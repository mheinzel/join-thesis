\documentclass[12pt]{article}

\usepackage[ngerman]{babel}
\usepackage[T1]{fontenc}
\usepackage[utf8]{inputenc}
\usepackage{hyperref}
\usepackage{lmodern}


\title{
  {\large Expos\'e zur Bachelorarbeit} \\
  Implementing the Distributed Join Calculus with Actors
}
\author{Matthias Heinzel\\Matrikelnummer 359143}
\date{\today}


\begin{document}
\maketitle

Prozesskalküle erlauben eine formale, abstrakte Beschreibung von nebenläufigen und verteilten Systemen, die mithilfe algebraischer Gesetze umgeformt und analysiert werden kann.
Sollen so beschriebene Prozesse jedoch ihrer Semantik getreu implementiert und verteilt ausführbar gemacht werden, ergeben sich oft Probleme.
So erfordern sowohl im synchronen als auch asynchronen Pi-Kalkül vorhandene Kommunikationsprimitive das Lösen komplexer Konsensprobleme, da sich Kanäle an keinem eindeutigen Ort befinden.

Sehr gut verteilen lässt sich hingegen der Join-Kalkül, bei dem sich alle Empfänger auf demselben Kanal immer auch am selben Ort befinden\cite{fournet_reflexive_1996}.
In seiner grundlegenden Definition ermöglicht er, asynchrone Nachrichten zu senden und mithilfe von Join-Patterns zu empfangen.
Zusätzlich existieren Erweiterungen wie Möglichkeiten zur synchronen Kommunikation, die mithilfe von Continuation-Passing auf die grundlegende Semantik zurückgeführt werden kann, explizite Verwaltung von Locations\cite{fournet_calculus_1996} sowie eine Typtheorie im Stil von ML\cite{fournet_implicit_1997}. Dadurch eignet er sich zur Einbettung in Programmiersprachen.

Es existieren bereits Implementierungen des Join-Kalküls in Form von Compiler-Erweiterungen (JoCaml\cite{mandel_programming_2007}, Polyphonic C\#\cite{benton_modern_2002}, Join Java\cite{itzstein_implementing_2003}) und Libraries (ScalaJoins\cite{haller_implementing_2008}, Joins Concurrency Library für .NET\cite{russo_joins_2007}).
Die bisher einzige Umsetzung von expliziten Locations zur Beschreibung von verteilten Systemen findet sich jedoch in einer nicht mehr gepflegten Beta-Version des JoCaml-Compilers, die Limitationen unter anderem bei der Erkennung von Failures aufweist\cite{conchon_jocaml:_1999}.
Spätere Versionen unterstützen Locations und die zugehörigen Sprachprimitive nicht mehr.

Um den hohen Implementierungs- und Wartungsaufwand von Erweiterungen an Compiler und Laufzeitumgebung zu vermeiden, bietet es sich an, auf existierenden Möglichkeiten zur verteilten Kommunikation aufzubauen.
So stellen zum Beispiel Implementierungen des Aktorenmodells bereits viel Funktionalität zur Verfügung, durch die eine direkte Auseinandersetzung mit den Details des Netzwerkstacks erspart bleibt.
Nötig wäre somit nur eine auf hoher Abstraktionsebene stattfindende Übersetzung von Konstrukten des Join-Kalküls in das Aktorenmodell.

Im Rahmen dieser Bachelorarbeit soll die Möglichkeit einer solchen Übersetzung überprüft und darauf aufbauend ein Prototyp einer verteilten Implementierung entwickelt werden.
Dazu ist zuerst die Auseinandersetzung mit der Semantik des Join-Kalküls und des Aktorenmodells notwendig, die in einer formalen Beschreibung der Übersetzung resultieren soll.
Diese dient dann als Grundlage der Entwicklung auf Basis einer existierenden Implementierung des Aktorenmodells.
Hier ist unter anderem zu klären, welche Vor- und Nachteile eine Umsetzung als Library, eine externe Kompilation zu Quellcode oder möglichst einfache Integration in einen existierenden Compiler bieten.

\section*{Zeitplan}

\begin{tabular}{| p{3cm} | p{7cm} | p{3cm} |}
  \hline
    Aufgabe &
    Details &
    Kalenderwochen \\
  \hline
    Grundlagen &
    Literaturarbeit, Schreiben der Kapitel zu den verwendeten Kalkülen und existierenden Implementierungen &
    36, 37 \\
  \hline
    Theoretische \newline Arbeit &
    Übersetzung des Join-Kalküls in das Aktorenmodell inkl. Kapitel dazu &
    38, 39 \\
  \hline
    Anforderungen &
    Sammeln von Anforderungen an die Implementierung &
    40 \\
  \hline
    Entwurf &
    Wahl von verwendeten Technologien, Programmstrukturierung &
    41 \\
  \hline
    Implementierung &
    Programmieren &
    42, 43, 44 \\
  \hline
    Validierung &
    Entwicklung von Beispiel\-anwendungen, Fehlerbehebung &
    44, 45 \\
  \hline
    Schreiben &
    Schreiben von Kapitel zu Implementierung, Einleitung und Auswertung &
    46, 47 \\
  \hline
    Vollendung &
    Korrekturen, Layout, Pufferzeit &
    48, 49 \\
  \hline
\end{tabular}

\nocite{levy_results_1997}
\nocite{maranget_compiling_1998}

\bibliography{expose}{}
\bibliographystyle{unsrt}

\end{document}
